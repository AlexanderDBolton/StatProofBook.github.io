\documentclass[a4paper,12pt]{article}

%%% Packages %%%
\usepackage{fullpage}
\usepackage{amsmath}
\usepackage{amssymb}
\usepackage{gensymb}
\usepackage{url}
\usepackage{csquotes}
\usepackage{enumitem}
\usepackage{setspace}
\usepackage[utf8]{inputenc}
\usepackage[bottom,hang,flushmargin]{footmisc}

%%% Settings %%%
\setlength{\parindent}{0pt}
\frenchspacing
\urlstyle{same}
\MakeOuterQuote{"}
\setlist{nolistsep}
\setlist[itemize]{leftmargin=*}
\setlist[enumerate]{leftmargin=*}


\begin{document}


**Theorem:** Let $A$ and $B$ be two statements about [random variables](/D/rvar). Then, if $A$ and $B$ are [mutually exclusive](/D/exc), their [joint probability](/D/prob-joint) is zero:

$$ \label{eq:prob-exc}
p(A,B) = 0 \; .
$$


**Proof:** If $A$ and $B$ are [mutually exclusive](/D/exc), then the [probability](/D/prob) of their disjunction is the sum of the [marginal probabilities](/D/prob-marg):

$$ \label{eq:exc}
p(A \vee B) = p(A) + p(B) \; .
$$

The [law of marginal probability](/D/prob-marg) implies that

\begin{equation} \label{eq:prob-marg}
\begin{split}
p(A) &= p(A,B) + p(A,\overline{B}) \\
p(B) &= p(A,B) + p(\overline{A},B)
\end{split}
\end{equation}

where $\overline{A}$ and $\overline{B}$ are the complements of $A$ and $B$. The probability of the disjunction $p(A \vee B)$ can also be expressed as the probability of a disjunction of three mutually exclusive statements

$$ \label{eq:prob-exc-s1}
p(A \vee B) = p\left([A \wedge \overline{B}] \vee [\overline{A} \wedge B] \vee [A \wedge B] \right) \; ,
$$

such that the definition of exclusivity can be applied to give

\begin{equation} \label{eq:prob-exc-s2}
\begin{split}
p(A \vee B) &\overset{\eqref{eq:prob-exc-s1}}{=} p\left([A \wedge \overline{B}] \vee [\overline{A} \wedge B] \vee [A \wedge B] \right) \\
&\overset{\eqref{eq:exc}}{=} p(A,\overline{B}) + p(\overline{A},B) + p(A,B) \\
&= [p(A,\overline{B}) + p(A,B)] + [p(\overline{A},B) + p(A,B)] - p(A,B) \\
&\overset{\eqref{eq:prob-marg}}{=} p(A) + p(B) - p(A,B) \; .
\end{split}
\end{equation}

Since $A$ and $B$ are [mutually exclusive](/D/exc), we obtain:

\begin{equation} \label{eq:prob-exc-qed}
\begin{split}
p(A \vee B) &\overset{\eqref{eq:prob-exc-s2}}{=} p(A) + p(B) - p(A,B) \\
p(A \vee B) &\overset{\eqref{eq:exc}}{=} p(A \vee B) - p(A,B) \\
p(A,B) &= 0 \; .
\end{split}
\end{equation}


\end{document}