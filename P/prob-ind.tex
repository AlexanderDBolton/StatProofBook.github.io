\documentclass[a4paper,12pt]{article}

%%% Packages %%%
\usepackage{fullpage}
\usepackage{amsmath}
\usepackage{amssymb}
\usepackage{gensymb}
\usepackage{url}
\usepackage{csquotes}
\usepackage{enumitem}
\usepackage{setspace}
\usepackage[utf8]{inputenc}
\usepackage[bottom,hang,flushmargin]{footmisc}

%%% Settings %%%
\setlength{\parindent}{0pt}
\frenchspacing
\urlstyle{same}
\MakeOuterQuote{"}
\setlist{nolistsep}
\setlist[itemize]{leftmargin=*}
\setlist[enumerate]{leftmargin=*}


\begin{document}


**Theorem:** Let $A$ and $B$ be two statements about [random variables](/D/rvar). Then, if $A$ and $B$ are [independent](/D/ind), [marginal](/D/prob-marg) and [conditional](/D/prob-cond) probabilities are equal:

\begin{equation} \label{eq:prob-ind}
\begin{split}
p(A) &= p(A|B) \\
p(B) &= p(B|A) \; .
\end{split}
\end{equation}


**Proof:** If $A$ and $B$ are [independent](/D/ind), then the [joint probability](/D/prob-joint) is equal to the product of the [marginal probabilities](/D/prob-marg):

$$ \label{eq:ind}
p(A,B) = p(A) \cdot p(B) \; .
$$

The [law of conditional probability](/D/prob-cond) states that

$$ \label{eq:prob-cond}
p(A|B) = \frac{p(A,B)}{p(B)} \; .
$$

Combining \eqref{eq:ind} and \eqref{eq:prob-cond}, we have:

$$ \label{eq:prob-ind-qed-A}
p(A|B) = \frac{p(A) \cdot p(B)}{p(B)} = p(A) \; .
$$

Equivalently, we can write:

$$ \label{eq:prob-ind-qed-B}
p(B|A) \overset{\eqref{eq:prob-cond}}{=} \frac{p(A,B)}{p(A)} \overset{\eqref{eq:ind}}{=} \frac{p(A) \cdot p(B)}{p(A)} = p(B) \; .
$$


\end{document}